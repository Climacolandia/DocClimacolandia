\section{Introdução}\label{intr}

  \subsection{Finalidade}
  \subsection{Escopo}
  \subsection{Visão Geral}

\section{Contexto da Empresa}\label{cont}

  \subsection{Organização da Empresa}
  \subsection{Descrição do Problema}

\newpage

\section{Justificativa da Abordagem}\label{just}

  \subsection{Características de análise da abordagem}
    \subsubsection{Formação do Processo}
  \subsection{Proposta de Abordagem}

\newpage

\section{Processo de Engenharia de Requisitos}\label{proc}

  \subsection{Metodologia de Desenvolvimento}
  
  A metodologia usada será o Incremental Commitment Spiral Model (ICSM). Em decorrência do baixo nível 
  de conhecimento técnico dos Stakeholders, a escolha desse processo de desenvolvimento foi realizada 
  com base, não somente por esse fator, mas também no seu alto nível de abstração que proporciona 
  grandes adaptações no projeto para fácil entendimento dos clientes. Além disso a metodologia funciona 
  como uma espécie de metodologia híbrida, pois traz conceitos de outras para formar a mesma.

  O ICSM une as fases do RUP, à flexibilidade do Ágil e a tomada de decisões baseada em riscos do 
  Espiral. A imagem 1 apresenta o funcionamento do processo em questão:

  % colocar modelo do icsm

  Cada ciclo da espiral representa uma fase do processo tradicional e para passar para a próxima fase 
  é preciso analisar os riscos de fazê-lo. É possível, como já foi dito, adaptar o processo para as
  necessidades do cliente, assim como o Ágil.

  \subsection{Modelagem do processo de Engenharia de Requisitos}

  O modelo utilizado neste tópico é baseada no Incremental Commitment Spiral Model (ICSM) que por
  acaso, como já definido neste documento, é um modelo de processo híbrido não muito conhecido, 
  com isso, pode-se trazer uma nova proposta de mistura de metodologias. Com essa característica, 
  somada às principais atividades da Engenharia de Requisitos: elicitar; documentar; analisar; 
  verificar e validar; e gerenciar, mais as fases definidas dentro do processo escolhido (ICSM), 
  exploração, valoração, fundamentação, desenvolvimento, operação e processo, onde alguns desse 
  processos são repetidos ao longo do desenvolvimento completo do sistema, pôde-se construir um 
  modelo de processo de requisitos mais flexível, adaptável às necessidades do clientes, e de modo 
  geral, diferente de modelos convencionais em vista dos aspectos gerais, como a ligação e 
  sequência estabelecida de atividades, sequência de entrega de artefatos, escolhas de troca de ciclo, 
  que no caso é baseada em riscos, etc (BOEHM, 2014). Esses aspectos se dão pelo fato de ser um 
  modelo criado a pouco tempo. 
  
  No diagrama abaixo, foi buscada a melhor visualização para todas as atividades e artefatos de 
  modo que se possa perceber a organização de todos os níveis e ligações entre atividades, 
  podendo assim, proporcionar a melhor análise e percepção, não apenas das mudanças de ciclos 
  mas principalmente das diferenciação da parte de análise de negócios e análise de sistema.

  % colocar imagem do processo

  Na figura 4, podemos observar várias estruturas de decisões que consistem em análise de risco, que 
  é a base da metodologia ICSM, com isso, para haver a passagem de um ciclo da espiral para o 
  próximo, ocorre a avaliação dos riscos que se tem ao realizar tal atividade. Dependendo do resultado 
  da análise, o projeto pode ficar estagnado ou até mesmo ser cancelado (BOEHM, 2014), portanto, 
  o prosseguimento da estrutura de decisão dependerá sempre do retorno vindo de determinada análise.
  
  Segundo (Boehm et. al.), os ciclos da espiral são divididos em: 

  \begin{\begin{itemize}
    \item \textbf{Exploração:} consiste apenas no entendimento do problema
    \item \textbf{Valoração:} Consiste em uma análise de custo benefício para avaliar a viabilidade do projeto
    \item \textbf{Fundamentação:} é onde ocorre a tomada de decisões a respeito de assuntos de ciclo 
    de vida do sistema, arquitetura, definição de tecnologias e semelhantes
  \end{itemize}}

  Esses três ciclos determinam a primeira fase, chamada de “Definição Incremental”, e pode levar de uma 
  semana a cinco anos, dependendo das características do projeto. No caso do projeto em questão, pretende-se 
  levar em torno de duas semanas na realização da primeira fase.
  
  Ainda tomando como base (BOEHM, 2014), após a fase de definição incremental, se inicia a fase 
  de “Desenvolvimento Incremental”. Nessa fase, o primeiro ciclo que ocorre está relacionado ao 
  desenvolvimento incremental e ao ajuste da fundamentação de acordo com as necessidades do cliente. 
  No próximo ciclo, é apenas adicionado a tarefa de operação e produção incremental enquanto as outras 
  duas se repetem. Os ciclos seguintes são repetições do anteriormente descritos.

  \subsection{Subprocessos da Engenharia de Requisitos}
    \subsubsection{Gerência de mudança}
    \subsubsection{Entrega dos casos de uso}
  \subsection{Papéis do Processo de Engenharia de Requisitos}
  \subsection{Artefatos do Processo de Engenharia de Requisitos}
  \subsection{Atividades do Processo de Engenharia de Requisitos}
    \subsubsection{Exploração e Valoração}
    \subsubsection{Fundamentação}
    \subsubsection{Desenvolvimento}
    \subsubsection{Operação e Produção}
  \subsection{Relação com o modelo de maturidade}

\newpage

\section{Elicitação de Requisitos}\label{elic}
  
  \subsection{Técnicas de elicitação de requisitos}
    \subsubsection{Entrevista}
    \subsubsection{Prototipagem}
    \subsubsection{Caso de Uso}

\newpage

\section{Gerenciamento de Requisitos}\label{geren}

\newpage

\section{Planejamento do Projeto}\label{plan}

  \subsection{Cronograma}
  \subsubsection{Descrições de algumas atividades}