\section{Introdução}\label{intr}

  \subsection{Finalidade}
  \subsection{Escopo}
  \subsection{Visão Geral}

\section{Contexto da Empresa}\label{cont}

  \subsection{Organização da Empresa}
  \subsection{Descrição do Problema}

\newpage

\section{Justificativa da Abordagem}\label{just}

  \subsection{Características de análise da abordagem}
    \subsubsection{Formação do Processo}
  \subsection{Proposta de Abordagem}

\newpage

\section{Processo de Engenharia de Requisitos}\label{proc}

  \subsection{Metodologia de Desenvolvimento}
  
  A metodologia usada será o Incremental Commitment Spiral Model (ICSM). Em decorrência do baixo nível de conhecimento técnico dos Stakeholders, a escolha desse processo de desenvolvimento foi realizada com base, não somente por esse fator, mas também no seu alto nível de abstração que proporciona grandes adaptações no projeto para fácil entendimento dos clientes. Além disso a metodologia funciona como uma espécie de metodologia híbrida, pois traz conceitos de outras para formar a mesma.

  O ICSM une as fases do RUP, à flexibilidade do Ágil e a tomada de decisões baseada em riscos do Espiral. A imagem 1 apresenta o funcionamento do processo em questão:

  Cada ciclo da espiral representa uma fase do processo tradicional e para passar para a próxima fase é preciso analisar os riscos de fazê-lo. É possível, como já foi dito, adaptar o processo para as necessidades do cliente, assim como o Ágil.

  \subsection{Modelagem do processo de Engenharia de Requisitos}
  \subsection{Subprocessos da Engenharia de Requisitos}
    \subsubsection{Gerência de mudança}
    \subsubsection{Entrega dos casos de uso}
  \subsection{Papéis do Processo de Engenharia de Requisitos}
  \subsection{Artefatos do Processo de Engenharia de Requisitos}
  \subsection{Atividades do Processo de Engenharia de Requisitos}
    \subsubsection{Exploração e Valoração}
    \subsubsection{Fundamentação}
    \subsubsection{Desenvolvimento}
    \subsubsection{Operação e Produção}
  \subsection{Relação com o modelo de maturidade}

\newpage

\section{Elicitação de Requisitos}\label{elic}
  
  \subsection{Técnicas de elicitação de requisitos}
    \subsubsection{Entrevista}
    \subsubsection{Prototipagem}
    \subsubsection{Caso de Uso}

\newpage

\section{Gerenciamento de Requisitos}\label{geren}

\newpage

\section{Planejamento do Projeto}\label{plan}

  \subsection{Cronograma}
  \subsubsection{Descrições de algumas atividades}